\setcounter{step}{0}
%------------------------------------------
% information doc
\subsection{Veterníky}
\PrepTime{60}
\CookingTime{30}
\CookingTempe{250}
\TypeCooking{Pečenie}
\NbPerson{4}
\Image{0 0 430 430}{images/veterniky} %style 2
%------------------------------------------

\begin{ingredient}

\vspace{0.5cm}

\begin{subingredient}{Cesto}
	\item 300 ml voda
	\item 150g palmarín
	\item 150g hladká múka
	\item 4ks vajíčka
\end{subingredient}
\begin{subingredient}{Plnka 1}
	\item 1-2ks zlatý klas
	\item 750ml mlieko
	\item 1ks vanilkový cukor
	\item 3.5PL práškový cukor
	\item 125ml šľahačka
\end{subingredient}
\begin{subingredient}{Plnka 2}
	\item 10PL kryštálový cukor
	\item 500ml šľahačka
\end{subingredient}
\begin{subingredient}{Poleva}
	\item 2.5PL kryštálový cukor
	\item 25ml mlieko
	\item 125g práškový cukor
\end{subingredient}
\end{ingredient}%no space with \begin{recipe}
\begin{recipe}

\step{Vodu s palmarínom necháme zovrieť a do toho pridáme múku.}
\step{Miešame 3 minúty nech sa cesto odliepa od hrnca. Necháme vychladnúť.}
\step{Do cesta primiešame vajíčka po jednom.}
\step{Na plech vytvarujeme ozdobným vreckom.}
\step{Pečieme 10 minút na 250C a potom 20 minút na 170C.}
\step{Ešte horúce veterníky rozrežeme.}

\substep[Pripravíme karamel]{Do hrnca dáme cukor aj na Plnku 2 aj na Polevu.}
\substep{Časť karamelu pridáme do misky s cukrom na polevu, a zmiešame. (Ak je zmes hustá, pridáme \emph{kúsok} mlieka.)}
\substep{Zvyšok karamelu si odložíme a neskôr pridáme k vymiešaným šľahačkám na Plnku 2.}

\step{Urobíme puding na Plnku 1.}

\substep[Pripravíme šľahačku]{Vymiešame šľahačku na Plnku 1 aj na plnku 2}
\substep{Po vychladnutí časť šľahačky primiešame do pudingu.}
\substep{Do zvyšku primiešame karamel.}

\step{Naplníme Plnkou 1, Plnkou 2 a polejeme Polevou}

\end{recipe}

\begin{notes}

\end{notes}	